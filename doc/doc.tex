\documentclass[a4paper,12pt]{article}

\author{}
\date{}
\title{Filtration of Ideal Gas and Pentane
\(\left( C_5 H_{12} \right) \)}

\usepackage[margin=0.9in]{geometry}
\usepackage{graphicx}
\usepackage{float}
\usepackage{textcomp}
\usepackage{amsmath, amssymb}
\usepackage{siunitx}
\usepackage{subcaption}
\usepackage{multirow}
\usepackage{bm}
\usepackage{ulem}
\usepackage{xcolor}

\definecolor{grey}{HTML}{737373}
\definecolor{red}{HTML}{d9343f}
\definecolor{green}{HTML}{2e8a07}

\begin{document}
\maketitle

\section{The Physics}

We simulate 2d flow of ideal gas and \(C_5H_{12}\) through a
porous medium with the help of Darcy's law:
\begin{equation}
    \bm{v}_i = -\frac{1}{\mu_i} \hat K \cdot f_\alpha (s)
    \cdot \nabla P
\end{equation}
where \(\hat K\), the specific permeability.
It depends only on the geometry of the medium.
We assume isotropy of space, so K is a scalar.
\(\mu\) is the dynamic viscosity.

\(i\) - component.

\(\alpha\) - phase. (If we had multiple phases, then
it would be \(f_\alpha\))

As an approximation, \(f_\alpha (s) = s^2\) for the first 
component, and \(f_\alpha (s) = (1 - s)^2\) for the second.

The continuity equation for each component becomes:
\begin{equation}
    \varphi \frac{\partial \rho_i}{\partial t}
    + div (\rho_i \bm{v}_i) = 0
\end{equation}
where \(\rho_i = \frac{m_i}{V}\).

We use the Tait equation to relate liquid density to pressure:
\begin{equation}
    \frac{\hat{\rho} - \rho_0}{\hat{\rho}} = C \log_{10}
    \frac{B + P}{B + P_0}
\end{equation}
where \(C = 0.2105\),
\(\rho_0 = \frac{1}{67.28 \frac{m^3}{mol}}\),
\(P_0 = 0.1 MPa\), \(B = 35MPa\),
in the case of \(C_5H_{12}\).

Ideal gas equation of state:

\begin{equation}
    P = \frac{RT}{M} \hat{\rho}
\end{equation}

\section{Boundary and Initial Conditions}

On the first iteration, we set an initial
pressure and molar composition.
Then, we derive the velocities from the 
pressure gradient using Darcy's law and the
saturation from the densities
and molar composition.

All our BC are 2-nd order (thanks to ghost
    cells), except for the BC on velocity on the inlet
and outlet, which end up 1-st order.

\begin{figure}[H]
    \centering
    \includegraphics[width=0.5\textwidth]{img/diagram.pdf}
    \caption{Boundary Conditions}
    \label{fig:img-diagram-pdf}
\end{figure}

\subsection{Pressure}

\textbf{BC:}

\[
\begin{cases}
    P = P_{in} &\text{at } y = 0
    \text{ and } x \in [0, 1] \\
    P = P_{out} &\text{at } y = 2 \\
    \frac{\partial P}{\partial x} = 0 &\text{at }
    x = 0, 2 \\
    \frac{\partial P}{\partial y} = 0 &\text{at }
    y = 0 \text{ and } x \in [1, 2]
\end{cases}
\] 

\textbf{IC:}

\[
\begin{cases}
    P = P_{out} &\text{at outlet} \\
    P = P_{in}  &\text{at inlet} \\
    P = P_0     &\text{everywhere else}
\end{cases}
\] 

\subsection{Velocities}

\textbf{IC:} Darcy 1-st order.

\textbf{BC:}

\[
\begin{cases}
    u = 0 &\text{at } x = 0, 2 \\
    v = 0 &\text{at } y = 0 \text{ and } x \in [1, 2] \\
    {\color{red}\text{Darcy (1-st Order)}}
          &\text{at } y = 2 \\
    {\color{red}\text{Darcy (1-st Order)}}
          &\text{at } y = 0
    \text{ and } x \in [0, 1]\\
\end{cases}
\] 

\subsection{Density}

We derive the densities from the equations of state.

\subsection{Saturation}
    Boundary condition on the inlet as the molar
    composition \(\psi\):
    \[
        \frac{m_1}{m_2} = \frac{\psi}{1 - \psi}
        \frac{M_1}{M_2}
    \] 
    where \(M_1\) and \(M_2\) represent the molar mass 
    of each component.

    We can derive the densities from the equations of state,
    then we can find the saturation.

    \[
    \hat \rho_1 = \frac{m_1}{sV}, \qquad
    \hat \rho_2 = \frac{m_2}{(1 - s)V}
    \] 

    \[
    \frac{\hat \rho_1}{\hat \rho_2} = \frac{m_1}{m_2}
    \frac{1 - s}{s}
    = \frac{\psi M_1}{(1 - \psi) M_2}\frac{1 - s}{s}
    \] 

    \[
        s = \left( 
        \frac{\hat \rho_1 M_2(1 - \psi) }{\hat \rho_2 M_1 \psi}
    + 1 \right)^{-1}
    \] 

    \textbf{On outlet:}

    \[
    \frac{\partial \alpha}{\partial \bm{n}} = 0
    \] 

    Or,

    \[
    \frac{\partial s}{\partial \bm{n}} = 0 
    \] 

\section{Discretization Scheme}

We use the finite difference method (FDM).

\subsection{Continuity Equation}

Second order in space. First order in time.

\[
\frac{\partial \rho}{\partial t}
+ \frac{\partial \rho}{\partial x} u
+ \frac{\partial \rho}{\partial y} v
+ \frac{\partial u}{\partial x} \rho
+ \frac{\partial v}{\partial y} \rho = 0
\] 

\[
\varphi \frac{\rho^{n + 1}_{i, j} - \rho^n_{i, j}}{\Delta t}
+ \frac{\rho_{i+1, j}^n - \rho_{i-1,j}^n}{2\Delta x} u_{i,j}
+ \frac{\rho_{i, j+1}^n - \rho_{i,j-1}^n}{2\Delta y} v_{i,j}
+ \frac{u_{i+1, j} - u_{i-1,j}^n}{2\Delta x} \rho_{i,j}^n
+ \frac{v_{i, j+1}^n - v_{i,j-1}^n}{2\Delta y} \rho_{ij}^n = 0
\] 

\subsection{Darcy's Law}

Second order in space.

\[
    u_{i,j}^n = -\frac{K}{\mu} f_\alpha (s_{i, j}^n)
    \frac{P_{i + 1, j}^n - P_{i - 1, j}^n}{2\Delta x} \\
\] 

\[
    v_{i,j}^n = -\frac{K}{\mu} f_\alpha(s_{i, j}^n)
    \frac{P_{i, j + 1}^n - P_{i, j - 1}^n}{2\Delta y}
\] 

\section{Finding Pressure using Binary Search}

The function \emph{find\_pressure} takes as arguments
\(\rho_1\) and \(\rho_2\), which are defined as

\[
\rho_1 = \frac{m_1}{V}, \qquad
\rho_2 = \frac{m_2}{V}
.\] 

We try to find the zero of the following function, that
takes the pressure as an argument:

\begin{equation*}
    f(P) = \frac{\hat{\rho_2} - \rho_0}{\hat{\rho_2}}
    - C \log_{10} \frac{B + P}{B + P_0}
.\end{equation*}

Here, \(\hat{\rho_2} = \frac{m_2}{(1 - s)V}\).
In order to determine \(\hat{\rho_2}\), we first find
\(\hat{\rho_1}\) using the EoS, and from there, we are
able to find the saturation
\(s = \frac{\rho_1}{\hat{\rho_1}}\).
Lastly, we determine \(\hat{\rho_2} = \frac{\rho_2}{1 - s}\).

% function findPressure(; ρ̂₁::T, ρ̂₂::T, V::T,
%         tait::TaitEoS, igas::IdealGasEoS,
%         p::Parameters,
%         left=1e4, right=1e7, niter=50,
%         eps_x=1e-6) where T<:AbstractFloat
% 
%     function f(P::AbstractFloat)
%         ρ₁ = density(igas, P)
%         s = ρ̂₁ / ρ₁
%         ρ₂ = ρ̂₂ / (1 - s)
%         return (ρ₂ - tait.ρ₀) / ρ₂ - tait.C *
%             log10((tait.B + P) / (tait.B + tait.P₀))
%     end
% 
%     P = binarySearch(; f, left, right, eps_x)
%     ρ₁ = density(igas, P)
%     s = ρ̂₁ / ρ₁
%     return P, s
% end
% 
\section{Algorithm}

Euler method for discretization with respect to time.

Second order scheme in space, with the use of ghost cells. 

\begin{enumerate}
    \item Calculate densities for each component using EOS.
    \item Find the pressure and saturation with the
        help of binary search.
    \item Use Darcy's law to calculate velocities.
\end{enumerate}

\section{Units and Parameters}

\begin{table}[H]
    \centering
    \caption{Parameters for our simulation.}
    \label{tab:label}
    \begin{tabular}{| c | c |}
        \hline
        Temperature & 298 \(K\) \\
        \hline
        \(P_{in}\) & \(10^6 Pa\) \\
        \hline
        \(P_{out}\) & \(10^5 Pa\) \\
        \hline
        Porosity, \(\varphi\) & 0.7 \\
        \hline
        Specific Permeability, \(K\) & \(10^{-12}\) \\
        \hline
        Dynamic Viscosity of Ideal Gas, \(\mu_1\) &
        \(1.8 \cdot 10^{-5}\) \(Pa \cdot s\) \\
        \hline
        Dynamic Viscosity of Pentane, \(\mu_2\) &
        \(2,14 \cdot 10^{-4}\) \( Pa \cdot s\) \\
        \hline
        Molar Mass of Ideal Gas, \(M_1\) & 
        \(0.028\) \(\frac{kg}{mol}\) \\
        \hline
        Molar Mass of Pentane, \(M_2\) & 
        \(0.07215\) \(\frac{kg}{mol}\) \\
        \hline
        Molar Composition at Inlet, \(\psi\) &
        \(0.3\) \\
        \hline
    \end{tabular}
\end{table}

\section{TODO}

\begin{enumerate}
    \item \sout{Fix: \(\mu\) is different for each component.}

    % \item Add boundary conditions on \(s_{n + 1} = s_n\)
    %     in the borders.

    \item \sout{Calculate the velocities in the first 
        iteration with a first order scheme. 
    Then, everything is calculated as normal.}

    \item \sout{The boundary condition \(\frac{dv}{dn} = 0\) 
            is usually used when solving the Navier-Stokes
        equation. In the case of filtration with Darcy's
        equation, we can either set the velocities
        explicitly or derive them from the pressure
        gradient on the boundaries and from the Darcy's
    equation on the inside.}

    \item \sout{Create boundary condition on the inlet 
            as the molar
        composition:}
        \[
            \frac{m_1}{m_2} = \frac{\psi}{1 - \psi}
            \frac{M_1}{M_2},
        \] 
        \sout{where \(M_1\) and \(M_2\) represent 
            the molar mass of each component.

        This way, we are essentially giving a boundary
        condition on the saturation, since we can derive
    the densities from the equations of state.}

    \item \sout{Change order of indexing: column major
            storage in memory.}

    \item \sout{Use naming conventions consistent with
            Julia base.}

    \item Use Real instead of T in functions' parameters.

    \item Create modules.

    \item \sout{Change density convention on 
        'BC: Saturation'.}

    \item \sout{Bug: The density functions don't
    take into account the saturation!}


    \item Fix: When initializing the system, it
        turns out that the pressure doesn't converge
        with the gas EoS. (Although it aligns with the
        liquid EoS).

    \item Upgrade to predictor-corrector.

\end{enumerate}

\section{Conventions}

\begin{enumerate}

    \item Density:

        \[
            \rho_i = \frac{m_i}{V}, \qquad
            \hat{\rho_i} = \frac{m_i}{s_i V}
        .\] 

\end{enumerate}
\end{document}
